\documentclass[a4paper, 11pt]{article}

%% Language and font encodings %%%%%%
\usepackage[utf8]{inputenc}
\usepackage[T1]{fontenc}

%%%%%%% Page Size, Geometry and Margin Size %%%%%%%%%
\usepackage[a4paper]{geometry}
\usepackage{parskip}

%%%%%%% USEFUL PACKAGES %%%%%%%%%
\usepackage{enumitem}
\usepackage{amsmath}
\usepackage{amsfonts}
\usepackage{amssymb}
\usepackage[dvipsnames, svgnames, table]{xcolor}
\usepackage{graphicx}
\usepackage[format=hang]{caption}
\usepackage{subcaption}
\captionsetup{format=hang, labelfont=bf}
\usepackage{microtype}
\usepackage[french]{babel}
\usepackage[colorlinks=true, linkcolor=blue, urlcolor=blue, hyperfigures=true]{hyperref}

\begin{document}

%%%%%%%%%%%%%%%%%%%%%%%% TITLE PAGE %%%%%%%%%%%%%%%%%%%%%%%%
\begin{titlepage}
\newcommand{\HRule}{\rule{\linewidth}{0.5mm}} % horizontal line and its thickness
\center

~\\
\vspace{3cm}

\HRule \\[0.5cm]
{ \huge \bfseries SYST0002 - Introduction aux signaux et systèmes\\ ~\\ \Large Rapport de projet}\\[0.5cm]
\HRule

~\\[2cm]

\large

\vspace{5cm}

{\Large
\begin{flushleft}
\begin{tabular}{l|l}
Professeurs & Guillaume Drion \& Alessio Franci\\
Assistants & Julien Brandoit \& Julien Vanderheyden \\
Auteurs & \textcolor{red}{????} \& \textcolor{red}{????}\\
Matricules & S\textcolor{red}{XXXXXX} \& S\textcolor{red}{XXXXXX}
\end{tabular}
\end{flushleft}
}

\vfill

\begin{center}
\textsc{Année académique 2025 -- 2026}
\end{center}

\end{titlepage}

%%%%%%%%%%%%%%%%%%%%%%%%%%%%%%%%%%%%%%%%%%%%%%%%%%%%%%%%%%%%

\newpage

\section*{Partie 1 -- Modèle d'état, simulations et analyse de stabilité}

\subsection*{Caractérisation du système}

%VOTRE RÉPONSE POUR LE PREMIER ITEM DE LA QUESTION 1. CECI EST ÉVIDEMENT À RETIRER ET À REMPLACER PAR LA DYNAMIQUE DU BICOPTERE...

La dynamique du système est donnée par l'équation~\eqref{eq:example} :
\begin{equation}\label{eq:example}
\begin{cases}
\dot{x} &= 4t^2 u + \sin x\\
\dot{y} &= x
\end{cases}\,,
\end{equation}
Le système est non-linéaire et variant dans le temps. En effet, on observe que la fonction non-linéaire $\sin(\cdot)$ s'applique à la variable d'état $x$ dans l'équation de sa dynamique, ce qui rend le système non-linéaire. De plus, le facteur $t^2$ rend le système variant dans le temps.

\subsection*{Écriture de la dynamique suivant le formalisme de la théorie du contrôle}

...
\newpage

\section*{Partie 2 -- Introduction d'un contrôleur}

...

\newpage

\section*{Partie 3 -- Fonction de transfert et diagrammes de Bode}

...

\end{document}